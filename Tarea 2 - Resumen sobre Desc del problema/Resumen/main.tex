%\documentclass[letterpaper,aps,twocolumn,pre,nofootinbib]{revtex4}
%\documentclass[twocolumn]{article}
\documentclass[conference]{IEEEtran}
\usepackage[spanish]{babel}
\usepackage{amsmath,amssymb,amsfonts,amsthm}
\usepackage{graphicx}
%\usepackage{bbm}
\usepackage[utf8]{inputenc} % Caracteres en Español (Acentos, ñs)
\usepackage{url} % ACENTOS
\usepackage{hyperref} % Referencias
\usepackage{subfig}
\usepackage{lipsum}
\usepackage{balance}


%%%%%%%%%%%%%%%%%%%%%%%%%%%%%%%%%%%%%%%%%%%%%
% PARCHE PARA ELIMINAR LA FECHA DEL DOCUMENTO
% 
\usepackage{etoolbox}
\makeatletter
% \frontmatter@RRAP@format is responsible for the parentheses
\patchcmd{\frontmatter@RRAP@format}{(}{}{}{}
\patchcmd{\frontmatter@RRAP@format}{)}{}{}{}
%\renewcommand\Dated@name{}
\makeatother	
% FIN DEL PARCHE
% 
%%%%%%%%%%%%%%%%%%%%%%%%%%%%%%%%%%%%%%%%%%%%%

%%%%%%%%%%%%%%%%%%%%%%%%%%%%%%%%%%%%%%%%%%%%%
% PARCHE PARA PERMIRIR UTILIZAR BIBLATEX EN ESTA PANTLLA
%\PassOptionsToPackage{square,numbers}{natbib}
%\RequirePackage{natbib}  
%%%%%%%%%%%%%%%%%%%%%%%%%%%%%%%%%%%%%%%%%%%%%

\usepackage[backend=bibtex,sorting=none]{biblatex}
% Estas lineas permiten romper los hipervinculos muy largos !!!!
\setcounter{biburllcpenalty}{7000}
\setcounter{biburlucpenalty}{8000}
\addbibresource{references.bib}

% Actualiza en automático la fecha de las citas de internet a la fecha de la compilación del documento
\usepackage{datetime}
\newdateformat{specialdate}{\twodigit{\THEDAY}-\twodigit{\THEMONTH}-\THEYEAR}
\date{\specialdate\today}

% la sentencia \burl en las citas... 
\usepackage[hyphenbreaks]{breakurl}

\renewcommand\spanishtablename{Tabla}
\renewcommand\spanishfigurename{Figura}

%\usepackage{datetime}
%\newdateformat{specialdate}{\twodigit{\THEDAY}-\twodigit{\THEMONTH}-\THEYEAR}
%\newdateformat{specialdate}{\twodigit{\THEDAY}-\THEYEAR}
%\date{\specialdate\today}


\begin{document}
%%%%%%%%%%%%%%%%%%%%%%%%%%%%%%%%%%%%%%%%%%%%%
% Definitions
%
%
% Define your special symbols here
%
%%%%%%%%%%%%%%%%%%%%%%%%%%%%%%%%%%%%%%%%%%%%%

% use to set width of figures
\newcommand{\breite}{0.9} %  for twocolumn
\newcommand{\RelacionFiguradoscolumnas}{0.9}
\newcommand{\RelacionFiguradoscolumnasPuntoCinco}{0.45}


%%%%%%%%%%%%%%%%%%%%%%%%%%%%%%%%%%%%%%%%%%%%%
% End Definitions
%%%%%%%%%%%%%%%%%%%%%%%%%%%%%%%%%%%%%%%%%%%%%


%Title of paper
\title{Resumen \\ Definición del Problema.}

% Trabajo Individual
\author{\IEEEauthorblockN{José Luis Pérez Avila\IEEEauthorrefmark{1}}
% En caso de trabajos en equipo, poner a todos los autores en estricto ORDEN ALFABETICO
%\author{\IEEEauthorblockN{Michael Shell\IEEEauthorrefmark{1},
%Homer Simpson\IEEEauthorrefmark{1},
%James Kirk\IEEEauthorrefmark{1}, 
%Montgomery Scott\IEEEauthorrefmark{1} and
%Eldon Tyrell\IEEEauthorrefmark{1}}
\IEEEauthorblockA{\IEEEauthorrefmark{1}Maestría en Ingeniería\\
Universidad Politécnica de Victoria}
}


%\date{}

\maketitle

El planteamiento del problema (objetivos del estudio, las preguntas de investigación y la justificación) y las hipótesis consecuentes surgen en cualquier parte del proceso en un estudio cualitativo: desde que la idea se ha desarrollado hasta, incluso, al elaborar el reporte de investigación. El planteamiento del problema se desarrolla una vez que se ha concebido la idea a investigar y el investigador ha profundizado en el tema en cuestión. Este no es sino afinar y estructurar más formalmente la idea de investigación. En ocaciones pasar al planteamiento del problema puede ser inmediatamente, esto depende de cuan familiarizado esté el investigador con el tema a tratar. El planteamiento del problema llega a tener lugar en diferentes momentos de la investigación: 1. en este segundo paso que sigue a la generación de la idea de investigación, 2. durante el proceso de investigacón y 3. Al final del proceso investigativo. Como señala Ackoff (1967): \textit{Un problema correctamente planteado está parcialmente resuelto}; a mayor exatitud corresponden más posibilidades de obtener una solución satisfactoria. Los criterios para plantear un problema son: 1. EL problema debe expresar una relación entre dos o más variables, 2. El problema debe estar formulado claramente y sin ambigüedad como pregunta por ejemplo, ¿qué efecto?, ¿en qué condiciones...? ¿cuál es la probabilidad de...? ¿cómo se relaciona...con...? y 3. El planteamiento debe implicar la posibilidad de realizar una prueba empírica o una recolección de datos. 

Los \textbf{elementos que contiene el planteamiento del problema} son: 1. Objetivos de investigación, 2. Preguntas de investigación, 3. Justificación de la investigación, 4. Criterios para evaluar el valor potencial de una investigación, 5. Viabilidad de la investigación y 6. Consecuencias de la investigación. Con los \textbf{objetivos de la investigación} se establece qué pretende la investigación. Estos tienen que expresarse con claridad para evitar posibles desviaciones en el proceso de investigación y deben ser susceptibles de alcanzarse. Durante la investigación es posible que surjan nuevos objetivos adicionales, se modifiquen los objetivos iniciales o incluso se sustituyan. Las \textbf{preguntas de investigación} nos ayuda a presentar de manera directa y minimizando la distorición de la investigación. No siempre en las preguntas se comunica el problema en su totalidad, a veces se formula solamente el propósito del estudio, aunque las preguntas debe resumir lo que habrá de ser la investigación. Las preguntas pueden ser más o menos generales, como se mencionó anteriormente, pero en la mayoría de los casos es mejor que sean más precisas. Se pueden plantear una o varias preguntas y acompañarlas de una breve explicación del tiempo, el lugar y las unidades de observación del estudio. Además de los objetivos y las preguntas, es necesario  \textbf{justificación} el estudio exponiendo sus razones. Las investigaciónes se efectúan normalmente con un propósito definido y ese propósito debe ser lo suficientemente fuerte para hacerse la investigacón. Debe de explicarse el porque es conveniete y cuales son sus beneficios. Cuando una investigación resuelve un problema social, a construir una nueva teoría o generar preguntas de investigación puede llegar ser conveniente realizarla. Establecer una serie de criterios para evaluar la utilidad de un estudio propuesto, los cuales deben ser flexibles y de ninguna manera son exhaustivos. Algunas preguntas para desarrollar los \textbf{criterios de evaluación} son: 1. Conveniencia,¿Qué tan convenite es la investigación?, 2. Relevancia Social. ¿Cuál es su trascendencia para la sociedad?, 3. Implicaciones prácticas, ¿Ayudará a resolver algún problema, 4. Valor teórico, ¿Con la investigación, se llenará algún huevo de conocimiento?, 5. Utilidad metodológica, ¿La investigación puede ayudar a crear un nuevo instrumento para recolectar o analizar datos?. En la \textbf{viabilidad de la investigación} se deben de tomar en cuenta la disponibilidad de recursos financieros, humanos y materiales que determinarán, los alcances de la investigación. Aunque no sea con fines científicas, es necesario que el investigador se cuestione acerca de las \textbf{consecuencias de su estudio}. 

Las \textbf{hipótesis} indican lo que estamos buscando o tratando de probar y se definen como explicaciones tentativas del fenómeno investigado, formuladas a manera de proposiciones. El formular una hipótesis depende de dos factores esenciales: el enfoque del estudio y el alcance inicial del mismo. Se pueden tener mas de una hipótesis, no siempre tiene que ser verdadera la hipótesis. Este elemento propone tentativamente las respuestas a las preguntas de investigación, la relación entre ambas es directa e íntima. Las hipótesis relevan a los objetivos y las preguntas de investigación para guiar el estudio. Por ello, las hipótesis comúnmente surgen de los objetivos y las preguntas de investigación. Caracteristicas de la hipótesis: 1. debe referirse a una situación social real, 2. los terminos deben ser comprensibles, precisos y concretos, 3. la relación entre variables debe ser clara y verosímil, 4. los terminos y la relación planteada entre ellos debe ser observable y mediable y 5. las hipótesis deben estar relacionadas con técnicas disponibles para probarlas. 

\addcontentsline{toc}{section}{Referencias}
\printbibliography
%\balance



\end{document}













